\documentclass{article}
\usepackage[letterpaper, margin=1.25in]{geometry}
\usepackage{url} % Link bibliography links (avoid long links for clarity)
\usepackage{breakurl} % Break URLs over multiple lines
\usepackage{hyperref}
\usepackage{apacite}%Also use in Capstone
\usepackage{fancyhdr}
\usepackage{setspace}
\usepackage{lipsum}
\pagestyle{fancy}
\doublespacing
\fancyhead[OHR]{Samurin \thepage}
\fancyhead[HL]{Use of Linux in the Real World}
\cfoot{}
\title{Use of Linux in the Real World}
\author{Jacob Samurin}
\begin{document}
\maketitle
\thispagestyle{fancy}
\newpage
\begin{center}
	\LARGE\textbf{ABSTRACT}
\end{center}
\begin{singlespace}
	\textit{Background:} Linux is a free and open source operating system that is used by almost everyone whether they know it or not. It is used in almost all the technology used by people in their day-to-day lives. \textit{Objectives:} This Lit Review's purpose is to educate people on there use of Linux in the real world. \textit{Results:} You will come out of this with a greater knowledge in technology and how it works it the world. \textit{Discussion / Conclusion:} With the greater understanding you can better come to your own conclusion on whether Linux is a worth time investment for many of the company that back Linux and is's adoption.\\
\end{singlespace}
\textbf{Keywords:} \textit{Linux, Daily Use, Real World, Server, Supercomputer, Linux From Scratch, LFS}
\newpage
\begin{center}
	\LARGE\textbf{Use of Linux in the Real World}
\end{center}
\section{INTRODUCTION}
Linux is as operating system (OS) that's simular to Microsoft's Windows OS and Apple's macOS.
The main difference is that Linux is free and open source which means that anyone can install and use Linux without a license, charge, and with the ability to look and modify the source code.
This gives Linux users the ability to do anything they want with the OS, which you can't do on Windows or macOS.
\\ \\
This is all important because all of this also make Linux widely versatile and used throughout every thing you use like cellphones and other handheld electronics that rely on Real-Time (RT) Programming.
With this and many other devices that use Linux it's important to know how these things operate and work.
That's why educating people about Linux is important.
\section{METHODS}
My study was completed by looking through multiple databases looking for research on more general use of Linux in industry and government.
With little outcome from this I moved to a more technical database and rethought out my sub-questions to more general.
That would better connect to the idea of the real world uses of this OS.
after I refined my sub-questions I started to look more into these sub-questions and as a result found the exact sources I needed from 3 different databases: Jstor, INFOhio, and Cornell University's arxiv database.
\section{RESULTS}
\subsection{Why is Linux used in these cases?}
Overall the reasons why Linux is widely used is because of six main reasons: stability, price, performance, capability, light-weight, and extensible.
One of these reasons is the stability of Linux, stability is when the operating system (OS) doesn't crash or often fail this leads to better uptime and easier maintainability.
There are some examples on projects that mainly choose Linux because of its stability like web and email server 
\\ \\
Another reason for Linux is the price, since it's free that leads lots of people to use it.
Two major examples I found are the use of Linux in libraries where buying hundreds of Microsoft's Windows licenses in impractical.
The other examples are Linux being used in south Asian countries that can't afford the expensive Windows licenses but still have lots of talent int the space.
This is a major reason for almost anyone wanting to switch to Linux even if you're a maintainer in a library or a poorer country like in the South Asia.
\\ \\
The next reason is the performance, Linux it's self is small so most of the power you get from the hardware goes to whatever you are trying to run.
For instance when you start talking about supercomputers and other computer centered around computing large amount of data you need all the performance you can get from the hardware you have.
Some examples of this are the NASA supercomputers that were bought from IBM and having all that performance in Linux made IBM choose it. 
Also lots of servers run Linux and it because if something goes down then you know it doesn't have to do with the OS. 
\\ \\
% capability
One of the reasons is capability, especially in the server and supercomputer space the things that Linux offer just doesn't exist in any other OS.
Another space that needs Linux because of its capability is phones and other handheld devices.
This is because phones need a Real-Time(RT) operations and Linux has this built it. 
Another example of the capability of Linux are the computers being used in less well-equipped nations like those in South Asia.
Linux can run on almost any computer or laptop so in a less advanced community can use older computers and put Linux on it and let these computers be used for years longer then they were supposed to. 
\\ \\
% light-weight
Another major reason that Linux is used is because it's light-weight this means that it take very little to run and up keep.
So on larger machines with lots of resources this will keep the use of those resources to a bare minimum.
This is useful when using lots of computers in a network because this makes it easy to deploy and update on a large scale. 
\\ \\
% extensable
The final point for Linux is the extensibility which is the ability to be able to change aspects of the OS and programs, so they will be able to do everything that is needed for the exact problem.
This is important when it comes to supercomputers because they need to use as much of the processing power as possible. 
\subsection{How is Linux made and what are the components of it?}
The best way to describe Linux is that it's a layered cake.
A layered cake is built from the bottom up Linux is the same way it has multiple layers that control the computer in different ways.
The main parts of Linux and all operating Systems(OS) are the hardware, kernel, Application Programming Interface (API), User.
\\ \\
One of the layer's hardware also known as the Basic Input/Output System (BIOS) is the closes you can get to the computer.
The hardware is where all the power of the machine is going to be here, so going up from here makes it harder for computers to understand but easier for humans.
When you are writing for the hardware level you are mostly writing assembly which is a low level programming language which means that you don't have lots of the nice things in modern high level language.
\\ \\
The next layer is the kernel and this is Linux actually is.
The Linux that has been talked about in this paper is actually Linux/GNU because the Linux part is actually just the kernel the rest of the tools are the GNU basic utilities.
The kernel is the layer that give the C and Rust languages support these are what are called system languages because they are built in and are still limited in the tools they have.
\\ \\
The layer after the kernel is the APIs these allow for lots of languages.
This layer makes it easy for any programmers to built graphics and do lots of things because the way it's done is abstracted from the actual way that it's done.
The APIs also allow for higher level languages like Python, Java, JavaScript, and many more.
\\ \\
Finally, is the user, this is you and how you interact with your computer.
This is the desktop and all the programs you use are run in this layer.
Since this is the farthest from the hardware it makes this the hardest on the resources.
\subsection{How is Linux used in the supercomputer spaces?}
In the world of supercomputer you are trying to get everything you can out of your dollar and your hardware, so most supercomputer run Linux with no user interface.
Linux is used in this setting because it is both free as in freedom.
With Linux, you can modify it in any way you like that is what free as in freedom means and this is important in the supercomputer world because you don't want to be slowed down by you Operating System (OS) and with Linux you can take out all the pieces that you don't need out of the OS entirely.
It is also free with aspects of cost.
This makes Linux the must-have option of the super-computing space for servers.
\subsection{What kind of Linux is used for all these applications?}
In the world of the Linux OS there are many of different types that all have their own purpose and strengths.
An example is in the server space a common distribution (distro) is Red Hat Linux.
Red Hat Linux is known for being stable and easy to manage which explain why it is used in the serve world.
Another example of a distro being used for a specific purpose is MontaVista Linux which is being used for real-time operations.
This example is used all over the real-time operations that are done by this distro is translated to your phone or other devices needing a real-time element.
This is truly a place where there is a wide and open place with lots of options and all of these options have upsides and downsides.
In these examples it's very clear that they have things that these distros are good at but if you for example were to use Red Hat Linux for real-time operations it is possible, but that's not what it's built for, same would be true if were to try the same thing with MontaVista and a server it works but not well
That's why it's important to find the distro that works best for the outcome you want.
\subsection{Where do you interactive with Linux daily?}
Linux is every were when you go to a website on your phone, update your computer, play video games, look up the weather, search something on Google, and many more.
All of these you Linux maybe in just one place or maybe many for example when you play a video game it has to connect to a server that is running there software with Linux in the background to make everything work.
Linux runs everywhere you don't directly see like in a server or a supercomputer, but sometimes it is a little closer than you think, all android phones use Linux as it's backbone because of the great Real-Time capability that come from Linux, and it's kernel.
Almost all websites that run off any of the big server providers run Linux and so does Amazon, Google, YouTube, and many other websites because it is stable and consistent.
Also, whenever you access a server it is very likely that server runs Linux.
So whenever you do anything that isn't run Locally then it is also most likely that you are using Linux there as well.
\subsection{Are there other way that Linux is being used by people?}
There is a very small community of people who run Linux as a desktop OS like macOS or Windows.
This community is mostly programmers and people who are very serious about their personal privacy.
One side of this community, the programming side use Linux because of the fast nature of the OS compared to Windows.
Another reason why programmers use Linux is because the tools they use usually run on a server that runs Linux, but sometimes they need to run those tool locally.
If you're on Windows there's no real way for you to be able to do this, but on Linux it is very easy because it is identical to how you would do it on the server.
More reasons programmers use Linux is because it has everything for any language and any workflow.
\\ \\
The other side of the community more focused on privacy use Linux for many reasons as well.
One of the reasons is that it's open source, so there is no mischief happening behind the scenes because you can see everything that is happening.
For instance if you're running Windows and you have to update you don't actually know what is updating, but when you update on Linux you know everything that is being updated.
Another reason privacy focused people use Linux is because there are very few viruses and malware that is written for Linux because the user base is so small and knowledgeable that it is a waste of time to develop these attacks.
In general the people who use Linux have a good use case and Linux makes there lives better and easier.
\section{DISCUSSION}
Linux's effect the world and everyone interacts with it even without the Knowledge of doing it.
The OSes use in every thing from Android to Real-Time applications to supercomputers to servers means that everyone has been affected by this OS in one way or another.
In order to do almost anything on a web browser or anything in the cloud because that all of it's using Linux and you may not know that, but you don't have to, but you are interacting with it.
There are also people who use Linux on the desktop and these people have reasons and doing this makes their lives better.
So in general for everyone Linux affects you in more ways that you know and there's lots of parts of computer use that uses Linux without your Knowledge.
\newpage
\nocite{*}
\bibliography{ref/capstone}{}
\bibliographystyle{apacite}
\end{document}
